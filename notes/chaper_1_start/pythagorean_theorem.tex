\documentclass[UTF8]{ctexart}

\titte{杂谈勾股定理}
\author{张三}
\date{\today}

\bibliographystyle{plain} % 声明参考文献格式

% 以上为导言区
\begin{document}

  \maketitle % 排版题目信息
  \tableofcontents % 命令输出目录
  \section{勾股定理在古代}
    勾股定理(英语Pythagorean theorem)又称商高定理(公元前 12 世纪)、毕达哥拉斯定理、毕氏定理、百牛定理,是平面几何中一个基本而重要的定 理。% 汉字左右的空格会被忽略

    % 使用空行分段,空行只有分段作用,段前不用打空格
    勾股定理说明,平面上的直角三角形的两条直角边的长度(古称勾长、股长)的平方和等于斜边长(古称弦长)的平方。反之,若平面上三角形中两边长的平方和等于第三边边长的平方,则它是
    % 换行相当于空格,只是分隔作用
    直角三角形(直角所对的边是第三边)。
  \section{勾股定理的近代形式}
  \bibliography{math} % 提示从math数据库获取文献信息,来打印参考文献


\end{document}
