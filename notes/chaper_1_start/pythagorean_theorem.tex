\documentclass[UTF8]{ctexart}

\titte{杂谈勾股定理}
\author{张三}
\date{\today}

\bibliographystyle{plain} % 声明参考文献格式

% 以上为导言区
\begin{document}

  \maketitle % 排版题目信息
  \tableofcontents % 命令输出目录
  \section{勾股定理在古代}
    勾股定理(英语Pythagorean theorem)又称商高定理(公元前 12 世纪)、毕达哥拉斯定理、毕氏定理、百牛定理,是平面几何中一个基本而重要的定 理。% 汉字左右的空格会被忽略

    % 使用空行分段,空行只有分段作用,段前不用打空格
    勾股定理说明,平面上的直角三角形的两条直角边的长度(古称勾长、股长)的平方和等于斜边长(古称弦长)的平方。反之,若平面上三角形中两边长的平方和等于第三边边长的平方,则它是
    % 换行相当于空格,只是分隔作用
    直角三角形(直角所对的边是第三边)。

    \begin{thm}[勾股定理]
      直角三角形斜边的平方等于两腰的平方和
    \end{thm}

    巴比伦人得到的勾股数的数量和质量不太可能纯从测量手段获得。之后的毕达哥拉斯本人并无著作传世,不过在他死后一千年,
    % 添加脚注
    5世纪的普罗克勒斯给欧几里德\footnote{欧几里德,约公元前330--275年}
    的名著《几何原本》做注解时将最早的发现和证明归功于毕达哥拉斯学派.

    在中国,记载秦朝的算数书并未记载勾股定理,只是记录了一些勾股数。定理首次载于书面则是在成书于西汉但内容收集整理自公元前一千多年以来的《周髀算经》“荣方问于陈子”一节中:
    \begin{quote}% 它将环境中的内容单独分行,增加缩进和上下间距,突出引用部分
    \zihao{-5}% 字号选择(-5 就是小五号)
    \kaishu% 使用楷书
      若求邪至日者,以日下为句,日高为股,句股各自乘,并而开方除之,得邪至日。

      --《周髀算经》卷上之二
    \end{quote}

  \section{勾股定理的近代形式}
  \bibliography{math} % 提示从math数据库获取文献信息,来打印参考文献


\end{document}
