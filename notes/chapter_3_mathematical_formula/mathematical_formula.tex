\documentclass[UTF8]{ctexart}

\title{第四章数学公式}
\date{\today}

% 数学公式的相关宏包
\usepackage{amsmath}

\DeclareMathOperator\dif{d\!}

\begin{document}
  \maketitle
  \section{数学公式概说}
    普通文字: a+b,1+1

    行内公式: $a+b$ , $1+1$

    直接写:交换律是

    \[ a+b = b+a, \]
    如
    \[
      1+2=2+1=3
    \]

    自动编号的数学公式
    \begin{equation}
      a+b = b+a\label{eq:communtative}
    \end{equation}

    \subsection{宏amsmath}
      可以用输入汉字公式

      $\text{被减数} - \text{减数} = \text{差}$

      在普通文本中使用数学公式注意文本与数学公式的转换,比如行内公式标点处不能换行,列举多项公式应该分别放在数学环境中.
  \section{数学结构}
    \subsection{上标与下标}
      上标下标可以嵌套

        $A_i^k = B^k_i$\qquad
        $K_{n_i} = K_{2^i} = K^{n_i} = 2^{2^i}$\qquad
        $3^{3^{\cdot^{\cdot^{\cdot^3}}}}$
        $F_{(x)}' = f_{(x)}$
        $ A = 90^\circ$

      显示公式中,多数数学算子的上下标在上方或者下方

        \[
          \max_n f(n) = \sum_{i=0}^n A_i
        \]

      但是行内公式会放在角标位置$\max_n f(n) = \sum_{i=0}^n A_i$

      积分号仍然会在角标位置
        % 导言区 \DeclareMathOperator\dif{d\!}
        \[ \int_0^1 f(t) \dif t + \iint_D g(x,y) \dif x \dif y\]
\end{document}
