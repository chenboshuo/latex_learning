\documentclass[UTF8]{ctexart}


\title{第四章数学公式}
\date{\today}

% 数学公式的相关宏包
\usepackage{amsmath}

% 显示代码和结果
\usepackage{codeshow}

% 化学元素的多角标
\usepackage{mathtools}

\DeclareMathOperator\dif{d\!}

\begin{document}
\maketitle
\section{数学公式概说}
普通文字: a+b,1+1

行内公式: $a+b$ , $1+1$

直接写:交换律是

\[ a+b = b+a, \]
如
\[
1+2=2+1=3
\]

自动编号的数学公式
\begin{equation}
a+b = b+a\label{eq:communtative}
\end{equation}

\subsection{宏amsmath}
可以用输入汉字公式

$\text{被减数} - \text{减数} = \text{差}$

在普通文本中使用数学公式注意文本与数学公式的转换,比如行内公式标点处不能换行,列举多项公式应该分别放在数学环境中.
\section{数学结构}
\subsection{上标与下标}
上标下标可以嵌套

$A_i^k = B^k_i$\qquad
$K_{n_i} = K_{2^i} = K^{n_i} = 2^{2^i}$\qquad
$3^{3^{\cdot^{\cdot^{\cdot^3}}}}$
$F_{(x)}' = f_{(x)}$
$ A = 90^\circ$

显示公式中,多数数学算子的上下标在上方或者下方

\[
  \max_n f(n) = \sum_{i=0}^n A_i
\]

但是行内公式会放在角标位置$\max_n f(n) = \sum_{i=0}^n A_i$

积分号仍然会在角标位置
% 导言区 \DeclareMathOperator\dif{d\!}
\begin{codeshow}
  \[ \int_0^1 f(t) \dif t + \iint_D g(x,y) \dif x \dif y\]
\end{codeshow}

积分通常采用上下限(limits)的排版方式(使用),使用(limits)可以使上下标在正上方或正下方,使用(nolimits)会使角标在角上
\begin{codeshow}
  \[
    \iiint\limits_D \mathrm{d}f
     = \max \nolimits_D g
  \]
\end{codeshow}


化学元素
\begin{codeshow}
  % \usepackage{mathtools}
  $\prescript{2}{1}{H}$
\end{codeshow}

对于巨算符号,可以使用sideset
\begin{codeshow}
  \[ \sideset{_a^b}{_c^d}
  \sum_{i=0}^n A_i
  = \sideset{}{'}
  \prod_k f_i \]
\end{codeshow}

化学宏包(mhchem)可以用(\\ce)输入化学式,这里省略

\subsubsection{上下划线与花括号}
\begin{codeshow}
  $\overline{a+b} =
  \overline a + \overline b$ \\
  $ \underline a = (a_0,a_1,a_2, \dots) $
\end{codeshow}

amsmath 提供了在公式上加箭头的指令
\begin{codeshow}
  $\overleftarrow{a+b}$\\
  $\overrightarrow{a+b}$\\
  $\overleftrightarrow{a+b}$\\
  $\underleftarrow{a+b}$\\
\end{codeshow}

数学字母的重音标记和宽标记用法类似,只不过通常较短,不能任意伸长,
但对于单个字母更准确
\begin{codeshow}
  $\vec x = \overrightarrow{a+b}$
\end{codeshow}

花括号可以用\\overbrace和\\underbrace
\begin{codeshow}
  $\overbrace{a+b+c}^{\text{共3项}} =
   \underbrace{1+2+3+4}_{n}$
\end{codeshow}

\end{document}
